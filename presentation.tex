\documentclass[14pt]{beamer}

\usepackage[utf8]{inputenc}
\usepackage{amsmath}
\usepackage{amsfonts}
\usepackage{amssymb}
\usepackage{graphicx}
\usepackage{caption}
\usepackage[ngerman]{babel}
\usepackage{hyperref}

\usetheme[navline=false,footlinenumber,footlinesection]{Bremen}
\fbname{Physik / Elektrotechnik}
\fbnum{01}

\author{StugA Physik}
\title[Kryptoparty]{eine Kryptoparty!}



\begin{document}

\begin{frame}
    \titlepage
\end{frame}


%\section{Wozu Krypto?}
\begin{frame}
    \frametitle{Wozu Krypto?}
    \begin{itemize}
        \item Verschlüsselung
        \item Signierung
    \end{itemize}

\end{frame}

\begin{frame}
    \frametitle{und wie?}
    \begin{itemize}
        \item OpenPGP
        \item Auf jeden Betriebssystem verfügbar
    \end{itemize}

\end{frame}

\begin{frame}
    \frametitle{und wie genau?}
    \begin{itemize}
        \item Private und Public Keys
        \item Verschlüsseln mit Public Key des Empfängers
        \item Entschlüsseln nur mit Private Key
    \end{itemize}

\end{frame}

\begin{frame}
    \frametitle{Signieren}
    \begin{itemize}
        \item Signieren mit Private Key
        \item Prüfen der Signatur mit Public Key
    \end{itemize}

\end{frame}

\begin{frame}
    \frametitle{Was ist noch wichtig?}
    \begin{itemize}
        \item Private Key muss geheim bleiben!
        \item Mit Passphrase sichern.
        \item Entwürfe nicht unverschlüsselt speichern!
    \end{itemize}

\end{frame}

\begin{frame}
    Und jetzt:\\
    \Large{https://emailselfdefense.fsf.org/de/}
\end{frame}

\end{document}
